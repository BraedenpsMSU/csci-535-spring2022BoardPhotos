\documentclass{article}
\usepackage{../fasy-hw}
\usepackage{ wasysym }

%% UPDATE these variables:
\renewcommand{\hwnum}{1}
\title{Computational Topology, Homework 1}
\author{\todo{your name here}}
\collab{\todo{list your collaborators here}}
\date{due: 3 February 2022}

\begin{document}

\maketitle

This homework assignment should be submitted as a single PDF file to D2L.

General homework expectations:
\begin{itemize}
    \item Homework should be typeset using LaTex.
    \item Answers should be in complete sentences, and make sense without
        seeing the question.
    \item You will not plagiarize, nor will you share your written solutions
        with classmates.  (But, discussing the questions is highly encouraged).
    \item List collaborators at the start of each question using the
        \texttt{collab} command.
    \item Put your answers where the \texttt{todo} command currently is (and
        remove the \texttt{todo}, but not the word \texttt{Answer}).
    \item If you are asked to come up with an algorithm, you are
        expected to give an algorithm that beats the brute force (and, if possible, of
        optimal time complexity). With your algorithm, please provide the following:
        \begin{itemize}
            \item \emph{What}: A prose explanation of the problem and the algorithm,
                including a description of the input/output.
            \item \emph{How}: Describe how the algorithm works, including giving
                psuedocode for it.  Be sure to reference the pseudocode
                from within the prose explanation.
            \item \emph{How Fast}: Runtime, along with justification.  (Or, in the
                extreme, a proof of termination).
            \item \emph{Why}: Justify why this algorithm works.  At a minimum, I
                expect a statement of the loop invariant for each loop, or
                recursion invariant for each recursive function.
        \end{itemize}
\end{itemize}


\nextprob{Taks}
% \collab{if applicable, update collab list}
\collab{n/a}

Please do the following:
\begin{enumerate}
    \item Write this homework in LaTex.  (You can use this document as a
        starting point!)  Note: if you have not used LaTex before and this is an
        issue for you, please contact me.
    \item Update your photo on D2L to be a recognizable headshot of you.
    \item Sign up for the class slack group.
    \item Fill out the course survey (found on the syllabus).
\end{enumerate}

\nextprob{}
% \collab{if applicable, update collab list}

Please read (or skim through) Eugenia Cheng’s
\href{http://eugeniacheng.com/wp-content/uploads/2017/02/cheng-proofguide.pdf}{Quick Guide for Writing
Proofs}
and describe one thing
that you already do well in your academic writing / proof writing,
and one thing that you will work on improving
throughout during this class.  You can reference additional resources, but be
sure to provide full references, either as footnotes or as a proper bibtex
citation!

\nextprob{}
% \collab{if applicable, update collab list}

Follow the Inkscape tutorial found here:
\url{http://tavmjong.free.fr/INKSCAPE/MANUAL/html/SoupCan.html}.
Then, make an
inkscape figure illustrating any concept that you would like.
Include the (PDF) images of both the soup can and your own figure as
floating figures in
your final write-up for this problem.  Don't forget to add a captions (and to
reference the figures!

\nextprob{}
% \collab{if applicable, update collab list}
 Watch the short \href{https://youtu.be/XfWibrh6stw}{YouTube} video.
 There is a topological
 error in the example he gives with the letter `A'.  What are the topological equivalence
classes of the letters A seen in this video?  Feel free to add additional
items to your equivalence classes for the letter `A'.  (You don't need to prove
that these are the equivalence classes, but a brief justification is expected).

\nextprob{}
% \collab{if applicable, update collab list}
Use the definition of big-O notation to prove that $f(x)=n^2 + 3n +2$ is
$O(n^2)$.

\nextprob{}
% \collab{if applicable, update collab list}
Let $f \colon \R \to \R$ be a function.
The function $f$ is considered open if for all open sets $A \subseteq \R$,
$f^{-1}(A)$ is open in $\R$, where open is defined by the standard topology on
$\R$ (so, what we normally think of as open).  Let~$c \in \R$.  Consider the
function $f_c \colon \R \to \R$ defined by $f_c(x) = c$ for all $x \in \R$. Prove
that $f_c$ is continuous.

\nextprob{}
% \collab{if applicable, update collab list}

Consider a graph $G=(V,E)$. The \emph{eccentricity} of a vertex $v \in V$ is the
maximum distance from $v$ to any other vertex in $V$. (Note: the \emph{(graph) distance} between
$a,b\in V$ is the minimum length of a path from $a$ to~$b$; the \emph{length} of
a path in an unweighted graph is the number of edges in the path). The
following algorithm computes the ecentricity of a vertex in a graph.  For the
while loop, provide the loop invariant and prove that it is the loop invariant.
In this algorithm, $Q$ is a (minimum) priority queue.

\begin{algorithm}
    \caption{Eccentricity(G,v)}
    \begin{algorithmic}[1]
        \REQUIRE a connected graph $G=(V,E)$ such that $|V| \geq 2$; a vertex $v \in V$
        \ENSURE the eccentricity of $v$
        \STATE For each vertex, add an attribute $dist$ and set it to $\infty$.
        \STATE $v.dist \gets 0$.
        \STATE Create a priority queue $Q$ of vertices, where each
                vertex $w\in V$ has priority $w.dist$.
        \STATE $maxdist \gets \infty$
        \WHILE{$Q$ is not empty}
            \STATE $w \gets $Q.pop() \COMMENT{1.5in}{Since $Q$ is a priority queue,
                                                $w.dist \leq x.dist$ for all $x \in Q$}
            \STATE $maxdist \gets w.dist$
            \FOR{each edge $(w,x) \in E$ such that $x \in Q$}
                \STATE $x.dist \gets w.dist +1$
            \ENDFOR
        \ENDWHILE
        \RETURN $maxdist$
    \end{algorithmic}
\end{algorithm}

\end{document}

