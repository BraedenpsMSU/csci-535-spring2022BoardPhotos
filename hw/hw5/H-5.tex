\documentclass{article}
\usepackage{../fasy-hw}

%% UPDATE these variables:
\renewcommand{\hwnum}{3}
\title{Computational Topology, Homework \hwnum}
\author{\todo{your name here}}
\collab{\todo{list your collaborators here}}
\date{due: 28 April 2022}

\begin{document}

\maketitle

This homework assignment should be submitted as a single PDF file to D2L.

General homework expectations:
\begin{itemize}
    \item Homework should be typeset using LaTex.
    \item Answers should be in complete sentences, and make sense without
        seeing the question.
    \item You will not plagiarize, nor will you share your written solutions
        with classmates.  (But, discussing the questions is highly encouraged).
    \item List collaborators at the start of each question using the
        \texttt{collab} command.
    \item Put your answers where the \texttt{todo} command currently is (and
        remove the \texttt{todo}, but not the word \texttt{Answer}).
    \item If you are asked to come up with an algorithm, you are
        expected to give an algorithm that beats the brute force (and, if possible, of
        optimal time complexity). With your algorithm, please provide the following:
        \begin{itemize}
            \item \emph{What}: A prose explanation of the problem and the algorithm,
                including a description of the input/output.
            \item \emph{How}: Describe how the algorithm works, including giving
                psuedocode for it.  Be sure to reference the pseudocode
                from within the prose explanation.
            \item \emph{How Fast}: Runtime, along with justification.  (Or, in the
                extreme, a proof of termination).
            \item \emph{Why}: Justify why this algorithm works.  At a minimum, I
                expect a statement of the loop invariant for each loop, or
                recursion invariant for each recursive function.
        \end{itemize}
\end{itemize}


\textbf{Choose five of the following problems to solve}

\nextprob{Dunce Cap}
HE-CT Chapter IV, Question 5. Page 102.  Part (i) and (ii) are required.  Part
(iii) is optional, but encouraged.


\nextprob{Classifying 3-Manifolds}
% \collab{TODO - uncomment if your collaborators on this have changed
HE-CT Chapter V, Question 4. Page 124.

\nextprob{Water and Land on a Manifold}
HE-CT Chapter V, Question 8. Page 124.

\nextprob{Reeb Graph}
HE-CT Chapter VI, Question 7. Page 146.

Bonus: what is the corresponding vineyard?

\nextprob{Bottleneck Metric}

Prove that the bottleneck distance is a metric on the space of persistence
diagrams.

\nextprob{Nerve Lemma}

The Nerve Lemma has many different forms.
\begin{enumerate}[(i)]
    \item What is the Bj\"orner nerve lemma/theorem?  Describe this theorem in
        your own words.
    \item Give three variants of the Nerve lemma.
        (Note: here, you are expected to find these in the literature)
\end{enumerate}


\nextprob{Vineyards}
Make a small video of a side-by-side comparison of a changing filtered simplex
with the changing persistence diagram.  (Note: this is one way to visualize a
vineyard).  Example filtrations you can use:
\begin{itemize}
    \item A height filtration of a graph in the plane, where the direction is
        changing.
    \item Morph between two images (e.g., linearly interpolate between the two
        of them) and sublevel set filtration on it.
\end{itemize}

\nextprob{Hypothesis Testing}

Suppose you have two sets of persistence diagrams.  Describe how you can use a
permutation test to determine if there is a statistical significance between the
two sets.  Don't forget the citations if your solution is modeled after an
existing method.

Bonus: empirically test this!  Choose two sets of diagrams (e.g., sampled from a
circle versus from a disc), and run your test for significance!

\nextprob{Biography}

Choose a researcher in computational topology and write a history of their
academic contributions.  Tell the story of how their research evolved and what
their major contributions are to the field of computational topology.  The
academic history should be written for a technical audience.

\nextprob{Research Dive}
The \href{https://topology.ima.umn.edu/seminars}{AATRN} has an archive of
several talks per semester since Fall 2014.  Choose one of the talks and
watch it, then answer the following questions:
\begin{enumerate}[(a)]
    \item Summarize the talk in about 1/2 page.
    \item Choose a published paper by the speaker, and read it; let's
        call this Paper-1.  First,
        skim the paper to get a general idea of the paper, then try to
        read as much detail as you can.  Describe where you got lost,
        using up to one full page.
    \item Identify a paper that would help enhance your understanding of
        Paper-1; let's call this Paper-2.  This paper can perhaps be
        found as a reference in Paper-1, or from your mad googling
        skills. Read it. First,
        skim the paper to get a general idea of the paper, then try to
        read as much detail as you can.  Describe where you got lost in
        Paper-2,
        using up to one full page.
    \item Repeat this process one more time: identify Paper-3 that would
        help you resolve where you got lost in Paper-2. Skim the paper,
        then read it in as much detail as you can: where do you get lost
        here?  Does this process have an end?
\end{enumerate}

\nextprob{Mock Review}
Look through the list of accepted papers to \href{https://www.inf.fu-berlin.de/inst/ag-ti/socg22/socg.html}{SoCG 2022}.  Find a preprint
of one of the papers (most are on ArXiv) and put a reviewer hat on.  (Alternatively, you can choose
the paper that you summarized in H-2).  Look at the tips on reviewing found
here:
\url{https://github.com/compTAG/student-resources/tree/master/how-tos/reviews}
Provide a review of the paper you selected.

\end{document}
